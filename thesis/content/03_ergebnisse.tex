\chapter{Numerik}



\section{Mathematische Aspekte der \lvn}
Die eindimensionale Wellenfunktion $\Psi(x)$ eines Teilchens ist ein Vektor des unendlich-dimensionalen Hilbertraums $L^2(\mathbb{R})$ mit dem üblichen Skalarprodukt
\begin{align}
  \bra{\Psi}\ket{\Phi} = \int_{\mathbb{R}} \Psi^*(x)\Phi(x) \diff x \; .
\end{align}
Beschränken wir uns auf ein Rechengebiet $L$, so ist entsprechend $\Psi(x) \,\in\,L^2(L)$. Diskretisieren wir ferner das System, so wird der Hilbertraum endlichdimensional mit Dimension $N$. Dann ist die Dichtematrix in Gleichung \eqref{eq:lvn} eine Matrix der Form $\mathbb{C}^N \times \mathbb{C}^N$ und der Liouville-Operator ein "Superoperator" \cite{frensley2} der Form $(\mathbb{C}^N \times \mathbb{C}^N)\times(\mathbb{C}^N \times \mathbb{C}^N)$.
Letztlich wird numerisch gesehen $N^2$ der Anzahl Freiheitsgrade entsprechen und die \lvn wird wieder eine Matrix-Vektor-Gleichung sein. Dazu wird $u(x,y)$ nicht als Matrix, sondern als Vektor der Länge $N^2$ geschrieben.
\todo{Eigenschaften von B(x,y) und A. Hermitizität von $\mathcal{L}$ (frensley).}
