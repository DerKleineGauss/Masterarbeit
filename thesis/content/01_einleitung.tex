\chapter{Einleitung}
\label{chap:einleitung}
Die numerische Simulation elektrodynamischer Quantenprozesse ist ein hochinteressantes Forschungsgebiet, da solche Simulationen der Schlüssel zur Entwicklung ultraschneller elektrischer Bauteile sind. Die physikalische Grundlage bildet dabei die Quantenstatistik. Unter vereinfachenden Annahmen an das zu untersuchende System können numerisch lösbare Gleichungen abgeleitet werden. 

Ziel dabei ist es, die experimentellen Ergebnisse für bestimmte Bauelemente zumindest ansatzweise korrekt zu beschreiben, um daraus Vorhersagen für weitere Strukturen mit beispielsweise geänderter Materialkomposition zu treffen. Als Referenz dient dabei in dieser Arbeit die \ac{rtd}, welche auch detailliert in der Literatur \cite{wiedenhaus} beschrieben ist.

Eine solche Struktur lässt sich als offenes System beschreiben \cite{frensley}. Es findet ein Austausch von lokal erhaltenen Fermionen -- die Erhaltung wird durch eine lokale Kontinuitätsgleichung beschrieben -- mit der Umgebung statt. Letztere besteht dabei aus zwei Teilchen-Reservoirs. Das zu untersuchende System besitzt eine endliche Ausdehnung im Raum. Es muss im Fall einer angelegten Spannung folglich ein Strom durch die Oberfläche als Rand des Systems fließen. Diese Arbeit beschränkt sich auf eindimensionale Quantenstrukturen, für die also die interessante Physik in einer Dimension stattfindet. Dabei soll der Strom der Teilchen, also der quantenmechanische Transport beschrieben werden. Dies geschieht allgemein in verschiedenen Anwendungsfällen (Hydrodynamik, Aerodynamik, Elektronik, Neutronentransport, usw.) mit Hilfe einer die Dynamik beschreibenden Differentialgleichung. Verschiedene Herangehensweisen führen dabei zu verschiedenen Gleichungen, wie die \ac{lvn} oder die Wigner-Gleichung.

Viele Methoden zur Lösung der \ac{lvn}  (beispielsweise in \cite{frenslely2}, \cite{lukas1}, \cite{}) sowie der Wigner-Gleichung (beispielsweise in \cite{rossi1994monte}, \cite{ringhofer}, \cite{van2017efficient}) sind in der Literatur untersucht worden. Es stellt sich die Frage: Wozu ein weiteres Verfahren?

Staying in space domain, the one-particle density matrix $\rho(r,r')$ for the one-dimensional problem is object of interest. It contains full quantum-mechanical information whereas physical observables like particle-, current- and energy-density are obtained by a projection of the matrix onto its diagonal. Applying a coordinate transformation towards center-of-mass coordinates and performing a FT on $\rho$ w.r.t. the relative coordinate leads to the wigner function formalism. That said, the relative coordinate can be associated with the particle's momentum $p$ in phase space.

Multiplying the Wigner equation with 1, $p$ and $|p|^2/2m$ and integrating over $p$ gives three conservation laws: particle-, momentum- and energy-conservation, see e.g. \cite{gardner1998approximation}. We want to model the exact same behaviour for the space domain equation. By analogy an integration over the relative coordinate yields the conservation law. Now, for the $x$-discretization the idea is to make use of a numerical method that is intended to deal with conservation laws, the so called Discontinuous Galerkin (DG) method. Finally, the transient solution is obtained by making use of an explicit time stepping method exploiting the fact, that the associated mass matrix can be inverted easily.

A particular crucial point is that the boundary conditions are assumed to be given in momentum space as originally depicted by Frensley \cite{frensley2}. Hence the derivative operator $\partial_y$ needs to be diagonalized. Our first approach was to perform a two-dimensional DG discretization. However it turned out that there are at least two drawbacks. First we can't separate the problem of diagonalizing $\partial_y$ but instead need to perform a FT of a discontinuous function. This is very inaccurate if evaluations at the interfaces are incorporated where the function is ambiguously defined. Second a rectangular grid would be needed to apply a consistent FT for each node along $x$. However, by introducing a rectangular grid the basis functions would be given by a tensor product essentialy leading to a method of line approach.

Regarding this, a method of line approach is chosen focusing on the characteristic of the underlying partial differential equation as motivated above. For the $y$-direction we make use of common integration-based methods (see below).
