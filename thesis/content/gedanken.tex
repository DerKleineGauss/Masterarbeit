\chapter{Entwicklung}
\section{Herleitung der Liouville-von-Neumann Gleichung}

Die Liouville-von-Neumann Gleichung erhält man aus der Liouville-Gleichung wie folgt.
\begin{align}
  \partial_t \hat{\rho} &= \frac{i}{\hbar}\left[\hat{\rho} , H\right] \\
  \underbrace{\bra{x} \partial_t \hat{\rho} \ket{y}}_{\equiv \partial_t \rho(x,y)} &= \bra{x}\ket{\frac{i}{\hbar}\left[\hat{\rho} , H\right]y} \\
   &= \frac{i}{\hbar} \sum_k p_k ( \underbrace{\bra{x} \ket{\Psi_k}}_{\equiv \Psi_k(x)}\bra{\Psi_k} \ket{Hy} - \bra{x}\ket{H\Psi_k}\underbrace{\bra{\Psi_k} \ket{y}}_{\equiv \Psi_k^*(y)} )
\end{align}
Mit dem Hamiltonoperator für ein einzelnes Teilchen in einer Dimension in Ortsdarstellung
\begin{align}
  \left[ -\frac{\hbar^2}{2m}\frac{\partial^2}{\partial x^2} + V(x) \right] \Psi(x) = \bra{x}\ket{H\Psi} \equiv \mathcal{L}(x)\Psi(x)
\end{align}
folgt mit $H^{\dagger} = H$
\begin{align}
  \partial_t \rho(x,y) &= \frac{i}{\hbar} \sum_k p_k \left( \Psi_k(x)\mathcal{L}^*(y)\Psi_k^*(y) - \mathcal{L}(x)\Psi_k(x)\Psi_k^*(y) \right) \\
  &= \frac{i}{\hbar}  (\mathcal{L}^*(y) - \mathcal{L}(x)) \sum_k p_k \left( \Psi(x)\Psi^*(y) \right) \\
  &= \frac{i}{\hbar}  (\mathcal{L}^*(y) - \mathcal{L}(x)) \bra{x}\left(\sum_k p_k  \ket{\Psi_k}\bra{\Psi_k} \right)\ket{y} \\
  &= \frac{i}{\hbar}  (\mathcal{L}^*(y) - \mathcal{L}(x))\rho(x,y)
\end{align}
Wir definieren noch
\begin{align}
  \mathcal{L}(x,y) &\equiv \mathcal{L}(x) - \mathcal{L}^*(y)\\
   &= -\frac{\hbar^2}{2m}\left( \partial_x^2 - \partial_y^2 \right) + V(x) - V^*(y)
\end{align}
und erhalten die Liouville-von-Neumann Gleichung im Ortsraum
\begin{equation}
  \partial_t \rho(x,y) = \frac{1}{i\hbar} \mathcal{L}(x,y) \rho(x,y) \; .
\end{equation}

\section{Schwerpunkt- und Relativkoordinaten}
Wir führen die Schwerpunkt und Relativkoordinaten
\begin{align}
  &r \equiv \frac{x+y}{2} \qquad &q \equiv x-y \\
  \Leftrightarrow\qquad &x = r+\frac{q}{2} \qquad &y = r-\frac{q}{2}
\end{align}
ein. Die Ableitungen transformieren sich dabei gemäß
\begin{align}
  \partial_r \partial_q  &= \partial_r \left( \frac{\partial}{\partial x} \frac{\partial x}{\partial q} + \frac{\partial}{\partial y} \frac{\partial y}{\partial q}\right) \\
   &= \left( \frac{\partial}{\partial x} \frac{\partial x}{\partial r} + \frac{\partial}{\partial y} \frac{\partial y}{\partial r}\right) \left( \frac{\partial}{\partial x} \frac{1}{2} + \frac{\partial}{\partial y} \frac{-1}{2}\right)\\
    &= \left( \frac{\partial}{\partial x} 1 + \frac{\partial}{\partial y} 1\right) \left( \frac{\partial}{\partial x} \frac{1}{2} + \frac{\partial}{\partial y} \frac{-1}{2}\right)\\
   &= \frac{1}{2}\partial_x \partial_x - \frac{1}{2}\partial_x \partial_y + \frac{1}{2}\partial_y \partial_x - \frac{1}{2}\partial_y \partial_y \\
  &=  \frac{1}{2}(\partial_x^2 - \partial_y^2) \; ,
\end{align}
wobei im letzten Schritt der Satz von Schwarz genutzt wird. Damit ergibt sich der transformierte Liouville Operator zu
\begin{align}
  \mathcal{L}(r,q) = -\frac{\hbar^2}{m} \partial_r\partial_q + \underbrace{V\left(r+\frac{q}{2}\right) - V^*\left(r-\frac{q}{2}\right)}_{\equiv e B(r,q)}
\end{align}
Das Potential ist nun in eV anzugeben. Mit der Umbenennung
\begin{align*}
  \rho \longrightarrow u \\
  r \longrightarrow \tilde{x} \\
  q \longrightarrow \tilde{y} \\
  t \longrightarrow \tilde{t}
\end{align*}
lautet die LvN Gleichung nun
\begin{align}
  i\hbar\partial_{\tilde{t}} u(\tilde{x},\tilde{y})+\frac{\hbar^2}{m}\operatorname{div}(A\nabla u(\tilde{x},\tilde{y})) - e B(\tilde{x},\tilde{y}) u(\tilde{x},\tilde{y}) = 0
\end{align}

\section{Charakteristische Einheiten}
Wir führen folgende Skalierung ein, um Längen und Zeiten numerisch zu behandeln.
\begin{align}
  \left(\begin{array}{c}\tilde{x}\\\tilde{y}\end{array}\right) &= \chi^{-1} \left(\begin{array}{c}x\\y\end{array}\right) &  & \chi = \sqrt{\frac{me}{\hbar^2}} \\
  \tilde{t} &= \xi^{-1} t   & \xi = \frac{e}{\hbar}
\end{align}
Damit folgt
\begin{align}
  \partial_t u(\chi^{-1}x,\chi^{-1}y)+\operatorname{div}(A\nabla u(\chi^{-1}x,\chi^{-1}y)) - e B(x,y) u(x,y) = 0
\end{align}


\section{Wigner Function}
Es ist mit $\bra{x}\ket{\Psi} = \Psi(x)$
\begin{equation}
  P(x,p) \equiv \frac{1}{\pi\hbar} \int_{-\infty}^{\infty} \bra{x+y}\hat{\rho} \ket{x-y} \E{2ipy/\hbar} \diff y
\end{equation}
die Wigner-Funktion gleich der Wigner-transformierten des Dichteoperators $\hat{\rho}$. Die Wigner Transformation ist eine invertierbare Abbildung
\begin{align}
  W\; :\; L(\HR,\HR)  \rightarrow & \text{Phasenraum}^* \\
   \hat{G} \mapsto & g(x,p) = \int_{-\infty}^{\infty} \bra{x-s/2}\hat{G} \ket{x+s/2} \E{ips/\hbar} \diff s
\end{align}
