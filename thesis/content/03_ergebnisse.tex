\chapter{Numerik}

% frensley3 zu QTBM (S.11)
% This requires that appropriate boundary conditions be formulatedan dap pliedt o
% (35). Lent and Kirkner [18] have demonstrated how to do this in the context of a finite-element electron
% waveguide calculation, andthe ir approach is calledthe Quantum Transmitting Boundary Method (QTBM).
% In the continuous case, one derives the QTBM conditions by evaluating ψ andit s derivative ψ at xl and
% xr using (30). One then solves for the incident amplitudes al and ar in terms of ψ and ψ, and imposes
% the resulting expressions upon Schr¨odinger’s equation as inhomogeneous boundary conditions. Conditions
% of this type, in which a linear combination of the function and its derivative are specified, are known as
% Robbins conditions.This requires that appropriate boundary conditions be formulatedan dap pliedt o
% (35). Lent and Kirkner [18] have demonstrated how to do this in the context of a finite-element electron
% waveguide calculation, andthe ir approach is calledthe Quantum Transmitting Boundary Method (QTBM).
% In the continuous case, one derives the QTBM conditions by evaluating ψ andit s derivative ψ at xl and
% xr using (30). One then solves for the incident amplitudes al and ar in terms of ψ and ψ, and imposes
% the resulting expressions upon Schr¨odinger’s equation as inhomogeneous boundary conditions. Conditions
% of this type, in which a linear combination of the function and its derivative are specified, are known as
% Robbins conditions.

% Greens : R. Lake and S . Datta, Phys. Rev. B 45, 6670 (1992).

\section{Transfer Function Methode}
\label{sec:TFmethod}
% frensley3 S.9
% In practical calculations, the transmission matrix approach has proven to be less than satisfactory, because
% it is prone to arithmetic overflow.
