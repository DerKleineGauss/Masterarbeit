% \chapter{Berechnete Grundenergien der isotropen Spinleiter}
% \label{anh:tabellen}
% Im Folgenden werden die Ergebnisse aus Kapitel \ref{sec:grundzustand} ergänzt um die entsprechenden Werte für \(J_\parallel\in\{0,2\;\;0,4\;\;0,6\;\;0,8\}\). Dies dient dem vollständigen Vergleich gegenüber der Berechnung aus \cite{4}.
% \input{files/Tabelle_Jperp_0.2_texformat.tex}
% \input{files/Tabelle_Jperp_0.4_texformat.tex}
% \input{files/Tabelle_Jperp_0.6_texformat.tex}
% \input{files/Tabelle_Jperp_0.8_texformat.tex}

\chapter{Ortsbasis}
\label{sec:A_1}
Für ein einzelnes Teilchen beispielsweise ist letztere die Menge der Eigenvektoren des Orstoperators
\begin{equation*}
  X:D_X \rightarrow L^2(\mathbb{R}^3;\mathbb{C}) \, : \, \Psi \mapsto x\Psi
\end{equation*}
zu den (i.A. komplexen) Eigenwerten $x$. Hierbei ist $D_X$ die Domäne
\begin{equation*}
  D_X = \{\Psi \in L^2(\mathbb{R}^3) \, | \, x\Psi \in L^2(\mathbb{R}^3)
\end{equation*}
des selbstadjungierten Operators $X$. Mit anderen Worten ist für eine Teilchenzahl $N$ der Zustand $\ket{x}$ derjenige Zustand, für den der Aufenthaltsort jedes Teilchens exakt bekannt (mit $x$ zusammengefasst) und wegen der Unschärferelation der Impuls vollständig unbekannt ist.
%Hierbei ist $S(\mathbb{R}^3)*$ der Raum der komplexwertigen temperierten Distributionen.
% S^* ist Dualraum des "Schwartz-Raums", welcher wiederum in allen Sobolew-Räumen enthalten ist. (Wiki) Der Raum der Testfunktionen lässt sich stetig in den Schwartz-Raum einbetten und liegt in diesem dicht.
die $\ket{x}$ stellen als Kontinuum eine uneigentliche Basis dar. Für die Spurbildung ist folglich ein Integral über $D_X$ auszuführen, statt einer Summation.

\chapter{Integration der Fermi-Dirac-Statistik}
\label{sec:A_2}
Im Folgenden wird die Teilchendichte (Teilchenzahl pro Fläche) für bezüglich $x$- und $y$- Richtung freie Elektronen im großkanonischen Ensemble mit chemischem Potential $\mu$ hergeleitet. Dabei ist die Fermi-Dirac-Statistik gemäß Gleichung \eqref{eq:wigner_n} über alle möglichen Impulse $\vec{k}=(k_x,k_y,k_z)$ zu integrieren. Allerdings ist die Dispersionsrelation bzgl. $k_z$ nicht bekannt, weshalb diese Variable zunächst als Freiheitsgrad beibehalten wird. 
\begin{align*}
  \frac{\expval{N}}{A_{\perp}} &= 2\frac{1}{A_{\perp}}\sum_{\vb{k}}\frac{1}{1+\exp(\beta(\epsilon(\vb{k}) - \mu))} \\
    &= 2\frac{1}{A_{\perp}}  \frac{A_{\perp}}{(2\pi)^2} \sum_{k_z}\int_{-\infty}^{\infty} \diff k_y \int_{-\infty}^{\infty} \diff k_x \frac{1}{1+\exp(\beta(k_x^2 + k_y^2 + k_z^2)\frac{\hbar^2}{2m} - \mu)} \\
    &= \frac{2}{(2\pi)^2} \sum_{k_z} \int_0^{2\pi} \diff \varphi \int_0^{\infty} \diff k_{\perp} k_{\perp} \frac{1}{1+\exp(\beta(k_{\perp}^2 + k_z^2)\frac{\hbar^2}{2m} - \beta \mu)}
\end{align*}
Substitution von $\epsilon = (k^2_{\perp} + k_z^2)\frac{\hbar^2}{2m} - \mu$ und somit $\diff k_\perp = \frac{m}{\hbar^2 k}\diff \epsilon$ sowie der Zusammenhang $\td{}{x}\ln(1+\exp(-\beta x)) = -\beta / (1+\exp(\beta x))$ liefern
\begin{align*}
  \frac{\expval{N}}{A_{\perp}} &= \frac{4\pi}{(2\pi)^2} \frac{m}{\hbar^2} \sum_{k_z} \int_{\epsilon(0)}^{{\epsilon(\infty)}} \diff \epsilon \frac{1}{1+\exp(\beta\epsilon)} \\
    &= \frac{m}{\pi\hbar^2}\left( \frac{-1}{\beta}\right) \sum_{k_z} \left.\ln(1+\exp(-\beta(k_{\perp}^2 + k_z^2)\frac{\hbar^2}{2m} + \beta\mu))\right|_0^{\infty} \\
    &= \sum_{k_z} \frac{m}{\pi\hbar^2\beta} \ln(1+\exp(\beta(\frac{- k_z^2\hbar^2}{2m} + \mu)))
\end{align*}
Für lokal konstantes $\beta$ folgt daher in Übereinstimmung mit \cite{frensley2} die Beschreibung der Reservoire gemäß
\begin{align}
  f_{l,r} (k) = \frac{m}{\pi\hbar^2\beta} \ln(1+\exp(\beta(\frac{- k^2\hbar^2}{2m} + \mu_{l,r}))) \; .
\end{align}
