% \chapter{Berechnete Grundenergien der isotropen Spinleiter}
% \label{anh:tabellen}
% Im Folgenden werden die Ergebnisse aus Kapitel \ref{sec:grundzustand} ergänzt um die entsprechenden Werte für \(J_\parallel\in\{0,2\;\;0,4\;\;0,6\;\;0,8\}\). Dies dient dem vollständigen Vergleich gegenüber der Berechnung aus \cite{4}.
% \input{files/Tabelle_Jperp_0.2_texformat.tex}
% \input{files/Tabelle_Jperp_0.4_texformat.tex}
% \input{files/Tabelle_Jperp_0.6_texformat.tex}
% \input{files/Tabelle_Jperp_0.8_texformat.tex}

\chapter{Ortsbasis}
\label{sec:A_1}
Für ein einzelnes Teilchen beispielsweise ist letztere die Menge der Eigenvektoren des Orstoperators
\begin{equation*}
  X:D_X \rightarrow L^2(\mathbb{R}^3;\mathbb{C}) \, : \, \Psi \mapsto x\Psi
\end{equation*}
zu den (i.A. komplexen) Eigenwerten $x$. Hierbei ist $D_X$ die Domäne
\begin{equation*}
  D_X = \{\Psi \in L^2(\mathbb{R}^3) \, | \, x\Psi \in L^2(\mathbb{R}^3)
\end{equation*}
des selbstadjungierten Operators $X$. Mit anderen Worten ist für eine Teilchenzahl $N$ der Zustand $\ket{x}$ derjenige Zustand, für den der Aufenthaltsort jedes Teilchens exakt bekannt (mit $x$ zusammengefasst) und wegen der Unschärferelation der Impuls vollständig unbekannt ist.
%Hierbei ist $S(\mathbb{R}^3)*$ der Raum der komplexwertigen temperierten Distributionen.
% S^* ist Dualraum des "Schwartz-Raums", welcher wiederum in allen Sobolew-Räumen enthalten ist. (Wiki) Der Raum der Testfunktionen lässt sich stetig in den Schwartz-Raum einbetten und liegt in diesem dicht.
die $\ket{x}$ stellen als Kontinuum eine uneigentliche Basis dar. Für die Spurbildung ist folglich ein Integral über $D_X$ auszuführen, statt einer Summation.
